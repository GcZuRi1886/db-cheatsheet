\documentclass[10pt,a4paper,landscape]{article}

% Pakete für kompaktes Layout und deutsche Umlaute
\usepackage[utf8]{inputenc}
\usepackage[ngerman]{babel}
\usepackage[margin=1cm]{geometry}
\usepackage{multicol}
\usepackage{amsmath}
\usepackage{amsfonts}
\usepackage{amssymb}
\usepackage{array}
\usepackage{siunitx}
\usepackage{makecell}
\usepackage{graphicx}
\usepackage{pdfpages}

% Kompakte Darstellung
\setlength{\parindent}{0pt}
\setlength{\parskip}{0.5ex}
\setlength{\columnsep}{1cm}

% Überschriften anpassen
\usepackage{titlesec}
\titleformat{\section}{\normalfont\Large\bfseries}{\thesection}{1em}{}
\titlespacing*{\section}{0pt}{1ex}{0.5ex}
\titleformat{\subsection}{\normalfont\large\bfseries}{\thesubsection}{1em}{}
\titlespacing*{\subsection}{0pt}{0.5ex}{0.25ex}

% Kopf- und Fußzeile entfernen
\pagestyle{empty}

% Redefine section commands to use less space
\makeatletter
\renewcommand{\section}{\@startsection{section}{1}{0mm}%
                                {-1ex plus -.5ex minus -.2ex}%
                                {0.5ex plus .2ex}%x
                                {\normalfont\large\bfseries}}
\renewcommand{\subsection}{\@startsection{subsection}{2}{0mm}%
                                {-1explus -.5ex minus -.2ex}%
                                {0.5ex plus .2ex}%
                                {\normalfont\normalsize\bfseries}}

\renewcommand{\subsubsection}{\@startsection{subsubsection}{3}{0mm}%
                                {-1ex plus -.5ex minus -.2ex}%
                                {0.5ex plus .2ex}%
                                {\normalfont\small\bfseries}}
\makeatother

\def\ojoin{%
  \rule[-.02ex]{.25em}{.4pt}\llap{\rule[.45em]{.25em}{.4pt}}}
\def\leftouterjoin{\mathbin{\ojoin\mkern-5.8mu\bowtie}}
\def\rightouterjoin{\mathbin{\bowtie\mkern-5.8mu\ojoin}}
\def\fullouterjoin{\mathbin{\ojoin\mkern-5.8mu\bowtie\mkern-5.8mu\ojoin}}

\setlength{\parindent}{0pt}
\setlength{\parskip}{0pt plus 0.5ex}

\begin{document}

\begin{center}
  \Large\textbf{Datenbanken}
\end{center}

\begin{multicols*}{3}
  \setlength{\premulticols}{1pt}
  \setlength{\postmulticols}{1pt}
  \setlength{\multicolsep}{1pt}
  \setlength{\columnsep}{2pt}

\section*{Relationale Algebra}
\begin{tabular}{|>{\centering\arraybackslash}m{0.5\linewidth}|>{\centering\arraybackslash}m{0.4\linewidth}|}
\hline
Operation & Symbol \\
\hline
Selektion & $\sigma_{B}(R)$ \\
\hline
Projektion & $\pi_{A_1, A_2, \ldots, A_n}(R)$ \\
\hline
Vereinigung & $R \cup S$ \\
\hline
Bag-Concatenation & $R \sqcup S$ \\
\hline
Schnittmenge & $R \cap S$ \\
\hline
Differenz & $R - S$ oder $R \backslash S$ \\
\hline
Kartesisches Produkt & $R \times S$ \\
\hline
Natural Join & $R \bowtie_{B} S$ \\
\hline
Full Outer Join & $R \; \fullouterjoin_{B} \; S$ \\
\hline
Left Outer Join & $R \; \leftouterjoin_{B} \; S$ \\
\hline 
Right Outer Join & $R \; \rightouterjoin_{B} \; S$ \\
\hline
Umbenennung & $\rho_{S(B_1, B_2, \ldots, B_n)}(R)$ \\
\hline
AND-Verknüpfung & $\sigma_{B_1 \land B_2}(R)$ \\
\hline
OR-Verknüpfung & $\sigma_{B_1 \lor B_2}(R)$ \\
\hline
Negation & $\sigma_{\neg B}(R)$ \\
\hline
Duplikatelöschung & $\delta(R)$ \\
\hline
\end{tabular}
\begin{center}
  $B$: Bedingung, $A_i$: Attribut, $R$, $S$: Relationen
\end{center}
\end{multicols*}
\end{document}
